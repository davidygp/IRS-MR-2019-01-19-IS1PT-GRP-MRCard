%!TEX root = ../Report.tex
With the multitude of credit cards available in the market, each with their own unique benefits and requirements, many consumers have a hard time in choosing which credit cards to get. In fact, many consumers choose credit cards which do not maximise their rewards based on their spending habits.

What card(s) a person should get, and how many benefits a consumer can reap from a card depends largely on how they spend and utilise their card monthly. Most credit cards give benefits based on how much a person spends by categories. For example, some cards give more benefits for grocery online  as compared to others cards, so consumers who do a lot of grocery shopping would benefit more from those cards. And benefits can vary according to specific sub-categories  (e.g. Cash rebate is more at NTUC vs Cold Storage). Another consideration for suitable cards is the type of benefit that the card gives. Typical benefits consist of cash rebates, air miles, and rewards points. With so many factors and categories to keep track of, it is very troublesome for a person to pick a card which best suits his lifestyle.

There are many terms to indicate value, in our report we use \textit{cashback} and \textit{rebate} interchangeably.