With the income of working adults in singapore steadily rising over the years, many people are gaining access to credit cards, especially young working adults. The majority of adults nowadays own at least one or more credit cards, with many others planning to start using credit cards as well. Banks have also been actively coming up with more credit cards and trying and to get consumers to take them up.

There can be many advantages in having a credit card. One advantage is that credit card users can earn benefits in terms of rebates, air miles, and rewards. This is usually the main draw for people to use credit cards. However, not every card is suitable for everyone. Each card has its own requirements and rates, and whether the user can earn the benefits from the card largely depends on their lifestyle and spending habits. With many credit cards available from the banks in Singapore, it can be a time-consuming task to pick up a suitable credit card, and many people simply get cards where their potential benefits are not maximised.

As a group of 5 young working professionals, we felt that this was a very relevant issue. Hence, we came up with the idea of designing a recommendation system to recommend the most suitable credit card or saving account based on the applicant's personal background, spending habits and personal preferences.

For this project, we first set out to perform knowledge acquisition by interviewing a subject matter expert, and also conducting a survey. To build the system, we decided to utilise the Django web framework, for its ease of integration with the front-end user interface (done with HTML), and the back-end rules engine (PyKnow) that we used to perform rule-based reasoning.

Our team learned a lot in the process of working on this project. We got the chance to apply techniques (like knowledge acquisition and rule-based reasoning) that we learned in our lectures and workshops in a viable business application scenario, and also picked up technical skills which would surely prove useful in the future course of our work.
